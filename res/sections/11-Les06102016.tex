\section{Informations about course}

\begin{itemize}
  \item 3 parts: information and biology, practice, make sense of the genome
with a bit of machine learning.
  \item Exercises on the Moodle platform
  \item Oral exam
  \item 20 lessons
  \item In the Moodle platform you find all the note of the last year.
\end{itemize}

\section{Biology and Information Technology}

There is a strong link between life and information because life represents a
``natural'' mechanism by which information is able to organize itself.

\subsection{Life}
Life is not only in our planet, there is also extra terrestrial life.
Programs are a subset of life $programs \subseteq life$. A thing is living
when for a given input it produce an output.

Biological organisms contain a program and information. The information of 
biological organisms is contained in \textbf{DNA}.

Part of this information is used to encode proteins (in mammals protein coding 
sequences are less than 2 percent of the genome).

The genetic program is essentially a \textbf{self replicating program}.

\subsection{Basic concept about DNA}

Deoxyribonucleic acid (DNA) is a molecule that carries the genetic instructions used 
in the growth, development, functioning and reproduction of all known living organisms
 and many viruses. 

DNA and RNA are \textbf{nucleic acids}.

Most DNA molecules consist of \textbf{two biopolymer strands} coiled (\textit{spirali}) 
around each other to form a double helix.

The two DNA strands are termed polynucleotides since they are composed of simpler monomer 
units called \textbf{nucleotides}.

Each nucleotide is composed of one of four nitrogen-containing nucleobases—either 
cytosine (C), guanine (G), adenine (A), or thymine (T)—and a sugar called deoxyribose 
and a phosphate group. 

The nucleotides are joined to one another in a chain by covalent bonds between the sugar 
of one nucleotide and the phosphate of the next, resulting in an alternating sugar-phosphate 
backbone.

DNA stores biological information. 
The DNA backbone is resistant to cleavage (\textit{rottura}), and both strands of the 
double-stranded structure store the same biological information. 
This information is replicated as and when the two strands separate. 

A large part of DNA (more than 98 \% for humans) is \textbf{non-coding}, meaning that these 
sections do not serve as patterns for protein sequences.

The two strands of DNA \textbf{run in opposite directions} to each other and are thus 
antiparallel. 

Attached to each sugar is one of four types of nucleobases (informally, bases). 
It is the sequence of these four nucleobases along the backbone that encodes biological 
information.

RNA strands are created using DNA strands as a template in a process called \textbf{transcription}. 
Under the genetic code, these RNA strands are translated to specify the sequence of amino 
acids within proteins in a process called translation.

\subsection{Evolution}

Three billion years ago some life (biology) originated, and then after millions
of years modern life (bacteria, animals, \dots) came in. All of this information
it's in the DNA. With the \textit{cultural evolution} we have information going
out from DNA, for example the ability to invent tools like computing machines.

An important thing it's that computer can learn, and it's possible to produce
intelligent and self-conscious machines, that will be able to improve themselves,
creating a \textit{singularity}.

In 1953 scientist found that was possible to change DNA to produce different forms
of life. People didn't believe that DNA contain all the information.
But it make sense that the information it's make from 4 amino acid, because
information it's difficult to duplicate.

\paragraph*{Proteins} Proteins are little ``machines'' that are able to do lot
of works. The problem is: ``how do you encode DNA to make protein?'' This is a
\textbf{central dogma}. It was discovered that from DNA you go to RNA and
finally to Protein thanks to ribosomes.
The main problem was how you can have all the information with only
4 nucleotide.

\paragraph*{Transfer RNA} tRNA is a molecule that has an Anticodon that with
the help of a ribosome it searches for a matching RNA pattern. Every tRNA has a
proper amino acid associated, that when put together it will create a protein.

\subsubsection{PCR: Polymerize Chain Reaction}
PCR is a technique to amplify a single or a few copies of a piece of DNA across
several orders of magnitude, generating thousands to million of copies of that
particular DNA fragment.

\subsubsection{RNA evolution in vitro}
Some nice experiments were done in the 90's by Szostak at Harward, to produce
in vitro evolution of RNA molecules. In these case ``software'' and ``hardware''
are represented by the same physical term: RNA.

One or the first experiments of this kind was about the evolution of an RNA
molecule able to bind ATP.
