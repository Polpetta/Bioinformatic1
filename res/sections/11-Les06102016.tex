\section{Information about course}

\begin{itemize}
  \item 3 parts: information and biology, practice, make sense of the genome
with a bit of machine learning.
  \item Exercises on the Moodle platform
  \item Oral exam
  \item 20 lessons
  \item In the Moodle platform you find all the note of the last year.
\end{itemize}

\section{Biology and Information Technology}

There is a strong link between life and information because life represents a
"natural" mechanism by which information is able to organize itself.

\subsection{Life}
Life is not only on our planet, there is also extra terrestrial life.
Also programs are a subset of life $programs \subseteq life$. A thing is living
when for a given input it produce an output.

\subsection{Evolution}

Three billion years ago some life (biology) originate, and this after millions
of years originated modern life (bacteria, animals). All of this information
it's in the DNA. With the \textit{cultural evolution} we have information going
out from DNA, for example the ability to invent tools like computing machines.

An important thing it's that computer can learn, and it's possible to produce
intelligent and self-conscious machine, that will be able to improve themselves,
creating a \textit{singularity}.

In 1953 scientist found that was possible to change DNA to make different forms
of life. People didn't believe that DNA contain all the information.
But it make sense that the information it's make from 4 amino acid, because
information it's difficult to duplicate.

\paragraph*{Proteins} Proteins are little "machine" that are able to do a lot
of works. The problem it's how do you encode DNA to make protein? This is a
\textbf{central dogma}. It was discovered that from DNA you go to RNA and
finally to Protein thanks to ribosomes.
The main problem it was how you can have all the information with only
4 nucleotide.

\paragraph*{Transfer RNA} tRNA is a molecule that has an Anticodon that with
the help of a ribosome it search for a matching RNA pattern. Every tRNA has a
proper amino acid associated, that when put together it will create a protein.

\subsubsection{PCR: Polymerize Chain Reaction}
PCR is a technique to amplify a single of a few copies of a piece of DNA across
several orders of magnitude, generating thousands to million of copies of that
particular DNA fragment.

\subsubsection{RNA evolution in vitro}
Some nice experiments were done in the 90's by Szostak at Harward, to product
in vitro evolution of RNA molecules. In these case "software" and "hardware"
are represented by the same physical term: RNA.

One or the first experiments of this kind was about the evolution of an RNA
molecule able to bind ATP.
