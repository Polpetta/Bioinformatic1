\section{Approaches}
\subsection{Reductionistic approach}

Biology is generally studied by means of the reductionistic approach

A complex system is not the simple sum of its components: a list of genes
does not tell us how the whole organism works, even if we know the molecular
function of each gene.

This approach says that a organism is far too complex to understand everything
in one time. The redustionistic approach tries to understand everything by
"splitting" the apparatus in small parts. Since there are so many different
component we need to annotate this components using ontologies and databases.

The second part of this approach is to put these informations togheter to make
sense of the living organisms.

Holistic approaches (in biology we refer to "systems biology") try to put
all pieces together and disclose emergent phenomena.

\textbf{Emergence} is a phenomenon whereby larger entities arise through
interactions among smaller or simpler entities such that the larger entities
exhibit properties the smaller/simpler entities do not exhibit.

\section{Genome sequences}
The instruments for DNA sequencing produce relatively short fragments of
sequence (\textbf{reads}), 100 to 1000 bases long, depending on the technology.
For example:
\begin{itemize}
  \item 1 human genom $\sim3$ Gbp
  \item 1 run $\to$ $\sim120$ Gbp reads $\approx150$ bases each
\end{itemize}

The DNA has 4 bases.
Proteins have 20 amminoacids.

An example of DNA:
\begin{verbatim}
  5' AATCCG 3'
  3' TTAGGC 5'
       =
  5' CGGATT 3'
  3' GCCTAA 5'
\end{verbatim}
