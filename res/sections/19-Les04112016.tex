\section{Genomic combination}

\subsection{Sanger assembly: overlap-layout-consensus algorithm}

With the sanger assembly we have to:
\begin{enumerate}
  \item calculat\end{enumerate}e pairwise alignments of all fragments
  \item choose two fragments with the largest overlap
  \item merge chosen fragments
  \item repeat setp 2 and 3 until only one fragment is left
\end{enumerate}

The problem with this approach is that it takes a lot of time (the complexity
is esponential). How can we solve this problem?

\subsection{Shotgun fragmentation and assembly}
With a genom we can try to see every read as a node and try to read it as an
Hamiltonian path. But this lead to the \textit{Hamiltonian Problem}, that is a
very difficult problem.
Another solution is to implement an Eulerian path, that goes once over each
edge, and it's a simpler problem.
The Eulerian solution doesn't solve the problem of overlaps.

\subsubsection{Be Bruijn graphs and Eulerian path}
This is the best solution for now, and use a study published by the
matematician Be Bruijn.
Steps for the solution:
\begin{enumerate}
  \item In pratical terms, a shotgun library is made and it is sequenced at a
reasonably high coverage, for instance 60x
  \item The resulting fastq file is read once, and all the kmers are counted.
This process will produce a list of all the kmers and their frequency
  \item We can start from any kmer that was found at the expected frequency and
we can extend the resulting conting by moving one position at a time. At every
step we will need to choose one of the 4 possible kmers. This will be done by
looking at the list of kmers. If there are no repeats then only one of the four
kmers should be present in the list.
\end{enumerate}

What is the best lenght of kamers to have the best use? What happens with long
kmers?
