\section{Genetic and Physical mapping}

\subsection{N50 statistic}

\textbf{Contig}: A contig (from contiguous) is a set of overlapping DNA
segments that together represent a consensus region of DNA.

In bottom-up sequencing projects, a contig refers to overlapping sequence
data (reads); in top-down sequencing projects, contig refers to the
overlapping clones that form a physical map of the genome that is used
to guide sequencing and assembly.

Contigs can thus refer both to overlapping DNA sequence and to overlapping
physical segments (fragments) contained in clones depending on the context. \\


De novo genomic assembly leads to a great number of contigs and scaffolds,
using respectively read overlaps and mate-paired analysis.

The next question is how can we put everything together.


Firstly we should have some metrics to measure the extent of the assembly.

For this purpose we use the so called N50 statistics.

A contig N50 is calculated by first ordering every contig by length from
longest to shortest.

Next, starting from the longest contig, the lengths of each contig are
summed, until this running sum equals one-half of the total length of the
genome (or, if the length of the genome is not available, you can use the sum
of the length of all contigs in the assembly).

The N50 value of a given assembly is the length of the shortest contig in this
list.

Another way to see it, is that 50\% of the genome is assembled into contigs
that are at least N50 bases long.

Generally, high values of N50 indicate better assembly.

The scaffold N50 is calculated in the same fashion but uses scaffolds rather
than contigs.

Like N50 you can calculate N25, N90, or any other value of N, indicating
respectively that 25\%, 90\%, or any other percentage of the genome is
assembled into contigs longer than N25, N90, etc bases.

Scaffolds and contigs with only a single read or read pair - often termed
singletons - are frequently excluded from these calculations.

Alongside with N50 very often is reported a L50 value that indicates the
number of contigs included in the list.

For instance, L50=350 and N50=42000 would mean that 50\% of the genome
is assembled into 350 contigs longer or equal 42000 bases.

This is very confusing because N indicates the length and L indicates
the number.

In fact, in some reports the two values are inverted, but common sense usually
helps.

\subsection{Genetic maps}
Between generation we have crossing over of genetic code because we are diploid.

CentiMorgan is a unit of genetic distance that represents a 1\% probability of 
recombination during meiosis. The maximum value is 50\%.

\subsection{Binning: other methods for placing landmarks on a genome}

You merge a human cell with a mouse cell to pick up only cells with only one 
chromosome through \textit{PCR}.

We need to map the genes in the chromosomes.
