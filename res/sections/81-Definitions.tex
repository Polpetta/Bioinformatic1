\chapter{Useful definitions}

\begin{itemize}
  \item \textbf{Diploid cells} contain two complete sets (2n) of
chromosomes
  \item \textbf{Haploid cells} have half the number of chromosomes
(n) as diploid/they contain only one complete set of chromosomes
  \item \textbf{mRNA} (messenger RNA) contains the informations to encode
proteins for the translation process

At the beginning and end of the mRNA there are untranslated regions (UTR)

In RNA instead than T there is U, that is chemically very similar, though T is
chemically more stable and therefore more suitable to store information.

Translation starts at the first AUG, and termitates at the first stop codon
There are three stop codons: UAG, UAA, UGA
  \item \textbf{tRNA} is an adaptor molecule composed of RNA, typically
76 to 90 nucleotides in length, that serves as the physical link between the
mRNA and the amino acid sequence of proteins.

It does this by carrying an amino acid to the ribosome as directed by a
three-nucleotide sequence (codon) in a messenger RNA (mRNA).

As such, tRNAs are a necessary component of translation, the biological
synthesis of new proteins in accordance with the genetic code.
  \item An \textbf{Allele} is the variant form of a given gene
  \item \textbf{Homozygous} means having identical pairs of genes for any
given pair of hereditary characteristics.
  \item In diploid organisms, \textbf{heterozygous} refers to having two
different alleles for a specific gene.
  \item \textbf{Meiosis} is a specialized type of cell division that
reduces the chromosome number by half, creating four haploid cells, each
genetically distinct from the parent cell.

Meiosis occurs in all sexually reproducing single-celled and multicellular
eukaryotes
  \item \textbf{Introns and exons} are parts of genes.
Exons code for proteins, whereas introns do not. Exons are converted into
mRNA
  \item \textbf{Splicing} is the editing of the nascent precursor messenger
RNA (pre-mRNA) transcript.

After splicing, introns are removed and exons are joined together (ligated).

For nuclear-encoded genes, splicing takes place within the nucleus either
co-transcriptionally or immediately after transcription.

For those eukaryotic genes that contain introns, splicing is usually required
in order to create an mRNA molecule that can be translated into protein.
  \item \textbf{PCR} (Polymerase chain reaction) is a technique used in
molecular biology to amplify a single copy or a few copies of a piece of DNA
across several orders of magnitude, generating thousands to millions of copies
of a particular DNA sequence.

It is an easy and cheap tool to amplify a focused segment of DNA, useful for
such purposes as the diagnosis and monitoring of genetic diseases,
identification of criminals (in the field of forensics), and studying the
function of a targeted segment of DNA.
  \item \textbf{Transcription} is the process by which DNA is used as a
template to create mRNA
\end{itemize}
