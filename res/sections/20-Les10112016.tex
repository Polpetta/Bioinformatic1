\section{Mate palled libraries}

There are two types of coverage: sequence coverage and physical coverage.

The pair-end aren't the same as mate-pairs.
Pair-end reads can be easily obtained from the library clone: firstly the 
sequence from one end is obtained, then with the other primer we can obtain the 
sequence from the other end.

\paragraph*{Problem with NGS sequencing} The chemistry for library 
amplification does not permit to manage fragments of DNA longer than a few 
hundred bases.
\paragraph*{Mate pair libraries overcome the limits of pair end libraries} The 
trick is to do a circular DNA. At the end of this process you have a library 
with million of this pairs, and this help enormously the assembly of the genome.

\section{Detection of structural variation with mate pair libraries}

Something there are changing in the structural variations using mate-pair 
libraries.
How can we estimate where there is a structural variation? We can plot the 
structural forms and see if there are some point that aren't in the normal 
distribution. Mate pairs can also be useful to discover structural variation in 
a genome as compared to a reference genome.
